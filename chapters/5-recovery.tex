\section{Pol\'iticas Steal y Force para el Buffer Manager}
\begin{itemize}
  \item Steal frame: \textit{¿Puede un elemento $X$ en memoria ser escrito en el disco antes de que la transacción $T$ termine?}
  \begin{itemize}
    \item Steal: Se puede mover al disco tan pronto como sea posible.
    \item No-Steal: No se puede mover al disco hasta que se haga \texttt{COMMIT}.
  \end{itemize}
  \item Force page: \textit{¿Es necesario que una transacción haga \texttt{FLUSH} de todos los elementos modificados inmediatamente antes o después de un \texttt{COMMIT}?}
  \begin{itemize}
    \item Force: Debe ser movido al disco antes de hacer \texttt{COMMIT}.
    \item No-Force: Puede ser movido en cualquier momento, incluso después del \texttt{COMMIT}.
  \end{itemize}
\end{itemize}

No-Steal + Force es f\'acil de recuperar, pero tiene el peor rendimiento. Por otro lado, Steal + No-Force no es tan fácil de recuperar, pero entrega mejor rendimiento. Este \'ultimo es el m\'as usado.

\section{Log Manager}
El Log Manager lleva un registro de todas las acciones de las transacciones. Todas las transacciones pueden escribir en el log de manera recurrente, y el log va llenando secuencialmente una p\'agina en el buffer. Luego, cuando esta p\'agina se llena, se almacena en el disco (secuencialmente también).

Existen varios tipos de Log Manager, pero todos comparten en com\'un los siguientes tipos de entrada.
\begin{itemize}
  \item \texttt{< START $T$ >}
  \item \texttt{< COMMIT $T$ >}
  \item \texttt{< ABORT $T$ >}
  \item \texttt{< $T$ \textit{update} >}
\end{itemize}
\pagebreak

\subsection{Undo-Logging}
Aquí, \texttt{< T \textit{update} >} $\rightarrow$ \texttt{<\ $T$, $X$, $t$ >}, donde $t$ es el valor antiguo de $X$.

\subsubsection{Reglas}
\begin{itemize}
  \item Regla $U_1$: ``Si $T$ modifica $X$, el log record \texttt{<\ $T$, $X$, $t$ >} debe ser escrito en el disco antes de que $X$ sea escrito en el disco.''
  \item Regla $U_2$: ``Si $T$ hace \texttt{COMMIT}, el log record \texttt{< COMMIT T >} debe ser escrito en el disco justo después que todos los datos modificados por $T$ sean guardados en el disco.''
\end{itemize}

\subsubsection{Procedimiento}
Para recuperarse de una falla usando Undo-Logging, se debe leer el log desde el final hasta el inicio.
\begin{itemize}
  \item \texttt{< START $T$ >}: Ignorar
  \item \texttt{< COMMIT $T$ >}: Marcar $T$ como realizada.
  \item \texttt{< ABORT $T$ >}: Marcar $T$ como realizada.
  \item \texttt{<\ $T$, $X$, $t$ >}: Restituir $X := t$ en el disco si $T$ no fue realizada.
\end{itemize}

\subsubsection{Checkpoints}
Hacemos undo recovery, hasta que llegamos a un checkpoint. Una vez que se llega a un checkpoint, se asume que todo lo que est\'a antes est\'a en el disco, y que todo funcion\'o como corresponde, por lo que no es necesario seguir.


\subsection{Redo-Logging}
Aquí, \texttt{< T \textit{update} >} $\rightarrow$ \texttt{<\ $T$, $X$, $t$ >}, donde $t$ es el valor nuevo de $X$.

\subsubsection{Reglas}
\begin{itemize}
  \item Regla $R_1$: ``Antes de modificar cualquier elemento $X$ en disco, es necesario que todos los logs records est\'en almacenados en disco, incluido el \texttt{< COMMIT $T$ >} log record.''
\end{itemize}

\subsubsection{Procedimiento}
Para recuperarse de una falla usando Redo-Logging, se debe leer el log desde el inicio hasta el final.
\begin{itemize}
  \item Se identifican las transacciones que hicieron commit.
  \item Se hace un scan desde el inicio.
  \item Para cada log record \texttt{<\ $T$, $X$, $t$ >}, no se hace nada si $T$ no hizo \texttt{COMMIT}. Si $T$ logr\'o hacer \texttt{COMMIT}, se reescriben los cambios en el disco acorde al log.
  \item Para cada transacción que qued\'o incompleta, se escribe \texttt{< ABORT $T$ >}.
\end{itemize}

\subsubsection{Checkpoints}
Al usar checkpoints, se sabe que al leer \texttt{< END CKPT >}, se garantiza persistencia de las transacciones que hicieron \texttt{COMMIT}.


\subsection{Undo/Redo-Logging}
Aquí, \texttt{< T \textit{update} >} $\rightarrow$ \texttt{<\ $T$, $X$, $t_O$, $t_N$ >}, donde $t_O$ es el valor antiguo de $X$ y $t_N$ es el valor nuevo de $X$.

\subsubsection{Reglas}
\begin{itemize}
  \item Regla $U_1$: ``Si $T$ modifica $X$, el log record \texttt{<\ $T$, $X$, $t$ >} debe ser escrito en el disco antes de que $X$ sea escrito en el disco.''
  \item Regla $U_2$: ``Si $T$ hace \texttt{COMMIT}, el log record \texttt{< COMMIT T >} debe ser escrito en el disco justo después que todos los datos modificados por $T$ sean guardados en el disco.''
  \item Regla $R_1$: ``Antes de modificar cualquier elemento $X$ en disco, es necesario que todos los logs records est\'en almacenados en disco, incluido el \texttt{< COMMIT $T$ >} log record.''
  \item Regla $UR_1$: ``Antes de modificar cualquier elemento $X$ en disco, es necesario que el log record \texttt{<\ $T$, $X$, $t_O$, $t_N$ >} est\'e registrado en el log.''
\end{itemize}


\subsubsection{Procedimiento}
Para recuperarse de una falla usando Undo/Redo-Logging, se debe hacer redo de todas las transacciones que hicieron commit, desde el principio hasta el final del log y tambi\'en se debe hacer undo de todas las transacciones que quedaron incompletas, desde el final hasta el principio del log.



